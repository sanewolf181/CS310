%CS310 Progress Report
%Hasan Ali
%0907383
%November 2011

\documentclass[a4paper,onecolumn,oneside]{article}
\newcommand{\HRule}{\rule{\linewidth}{0.5mm}}
\usepackage{tabularx}
\usepackage{hyperref}
\usepackage{graphicx}

\begin{document}

\title{{\large A Computational Model of Chemotaxis in E.Coli} \\ [0.4cm] \textbf {Progress Report} }
\author{Hasan Ali}
\date{\today}
\maketitle

\begin{abstract}
In this progress report, I will show that the design phase for my project has been concluded,
that all forseeable issues have been overcome, and that my project is on track to be a success.
I will also detail any changes I have made to my specification in order to make my project realisable.
\end{abstract}

\section{Project Completion}
My project has been partially realised so far. I have created a basic, though not accurately modelled simulation of chemotaxis
behaviour in E.Coli. The graphical display needs considerable work, but does not pose any major issues.

\section{Research Conducted}
Because my project concerns a subject area which I am not very familiar with,  I have had to 
do significant research in order to fully comprehend the mechanics behind it, and to make certain that
my project is feasable. Over the course of the previous weeks, I have realised that my project concerns 
not only the detailed mechanical algorithms used by bacteria to move about, but also the mathematical models
of cellular automata, and the physical method of diffusion of a chemical. A more practical concern is
how to implement the algorithms efficiently as well as accurately, as my simulation must run at a reasonable
speed.

\subsection{John Conways Game of Life}
I have also researched John Conways game of life, which 
\subsection{History of previous research}
T.W.Englemann -- 1881
Dennis Bray -- university of cambridge -- 2006

\subsection{Significance of field in Computational Biology}
I have discovered that accurate simulations of chemotaxis have significant impacts on preventing the spread of disease.
\subsection{Use of Monte Carlo methods in Computational Biology}

\section{Challenges and Issues}
I have encountered a challenges in the preliminary implementations of the simulation. These mainly concerning how to make sure my simulation
accurately models the behaviour of E.Coli
\subsection{Graphical Interface Issues}
One of the challenges I will have is how to make my simulation run efficiently -- the preliminary model I have created redraws every cell in the simulation, 
including those which have not changed. I must rethink my implementation so only those that need are repainted.
\subsection{Mathematics behind the Simulation}
I had not considered the depth of the mathematical equations I would need to implement. After doing some research, I have found the the equations are in fact quite complex,
and will require greater time to accurately model.
\subsection{Dealing with Edge cells of the Simulation}
I had not thought of how to deal with the cellular automata at the border of my simulation; in fact at first it appeared I would need a different ruleset to those in the middle. However, 
after experimenting, I decided that creating an infinite grid of cells would be the best approach, thus forming a toroidal shaped plane. Intuitively, the borders simply wrap round, so going off the top of 
the grid simply brings you back at the bottom.
\subsection{Underestimated depth of biology,mathematics and organic chemistry required}
I underestimated how much of the project would be theoretical, and how much research would be required. \\
Thus I have rescheduled my timetable to allow time to gain a greater depth of knowledge, and to focus on an accurate simulation rather than superficial features\\
eg. I am going to focus on E.Coli, in line with previous research.
\subsection{New skills required to complete project}
\begin{itemize}
\item Learning Latex, a high quality typesetting markup language, in order to display mathematical formulae, and write a professional looking final report
\item Learning how to use research facilities available to me in the library, and online
\item furthering my experience with the java platform, in particular the use of swing and awt (abstract windowing toolkit)
\end{itemize}
\section{Cellular Automata}
My project seems to be split over two interesting areas - one the theory, and two the implementation. \\
The theory is the high level of mathematics I will have to go into to accurate model the behaviour of E.Coli. \\
The implementation is providing efficient algorithms that implement the behaviour, \\
creating the cellular automata is a significant part of the implementation. \\
\begin{itemize}
\item Squares chosen rather than hexagonal. partly for ease of implementation
\item how to deal with cells at borders
\item john conways game of life - my inspiration
\item how to accurately simulate diffusion of chemicals

\end{itemize}

\section{Changes to Specification}
In order to deal with unforseen challenges, I have had to make a few modification to my specification. These include scheduling changes, different implementation methods, and dropped objectives.

\begin{thebibliography}{9}

\bibitem{bbcnews1}
  BBC News,
  \emph{Computer bug study wins top prize},
  \url{http://news.bbc.co.uk/1/hi/sci/tech/6113522.stm},
  \date{3 November 2006}.

\bibitem{ecoli1}
  Diana Clausznitzer, Olga Oleksiuk, Linda Løvdok, Victor Sourjik, Robert G. Endresi, 
  \emph{Chemotactic Response and Adaptation Dynamics in Escherichia Coli},
  PLoS Computational Biology, Vol.6,
  \date{May 2010}

\bibitem{muCell1}
  Dominic Orchard, Jonathan Gover, Lee Lewis Herrington, James Lohr, Duncan Stead, Cathy Young and Sara Kalvala,
  \emph{µCell - Interdisciplinary Research in Modelling and Simulation of Cell Spatial Behaviour},
  Reinvent\'{i}on, Vol.2 Issue 1, \url{http://www2.warwick.ac.uk/fac/cross_fac/iatl/ejournal/issues/volume2issue1/orchard}, \date{30 April 2009}.
  
\end{thebibliography}
\end{document}
